%%%%%%%%%%%%%%%%% NO CAMBIAR ESTA PARTE %%%%%%%%%%%%%%%%%%%% 
%%%%%%%%%%%%%%%%%%%%%%%%%%%%%%%%%%%%%%%%%%%%%%%%%%%%%%%%%%{
    \documentclass[twoside,11pt]{article}
    %%%%% PACKAGES %%%%%%
    \usepackage{pgm2016}
    \usepackage{amsmath}
    \usepackage{algorithm}
    \usepackage[noend]{algpseudocode}
    \usepackage{subcaption}
    \usepackage[spanish, mexico]{babel}	
    \usepackage{paralist}	
    \usepackage[lowtilde]{url}
    \usepackage{fixltx2e}
    \usepackage{listings}
    \usepackage{color}
    \usepackage{hyperref}
    \usepackage{auto-pst-pdf}
    \usepackage{pst-all}
    \usepackage{pstricks-add}

    %%%%% MACROS %%%%%%
    \algrenewcommand\Return{\State \algorithmicreturn{} }
    \algnewcommand{\LineComment}[1]{\State \(\triangleright\) #1}
    \renewcommand{\thesubfigure}{\roman{subfigure}}
    \definecolor{codegreen}{rgb}{0,0.6,0}
    \definecolor{codegray}{rgb}{0.5,0.5,0.5}
    \definecolor{codepurple}{rgb}{0.58,0,0.82}
    \definecolor{backcolour}{rgb}{0.95,0.95,0.92}
    \definecolor{ucvia}{RGB}{21,110,182}
    \lstdefinestyle{mystyle}{
       backgroundcolor=\color{backcolour},  
       commentstyle=\color{codegreen},
       keywordstyle=\color{magenta},
       numberstyle=\tiny\color{codegray},
       stringstyle=\color{codepurple},
       basicstyle=\footnotesize,
       breakatwhitespace=false,        
       breaklines=true,                
       captionpos=b,                    
       keepspaces=true,                
       numbers=left,                    
       numbersep=5pt,                  
       showspaces=false,                
       showstringspaces=false,
       showtabs=false,                  
       tabsize=2
    }
    \lstset{style=mystyle}

    %%%%% DATOS INSTITUCIONALES %%%%%%
    \newcommand\university{Universidad Central de Venezuela}
    \newcommand\course{6213}
    \newcommand\courseName{Miner\'ia de datos}
    \newcommand\evaluation{Parcial}
    \newcommand\assignmentNumber{1}
%%%%%%%%%%%%%%%%%%%%%%%%%%%%%%%%%%%%%%%%%%%%%%%%%%%%%%%%%% 
%%%%%%%%%%%%%%%%%%%%%%%%%%%%%%%%%%%%%%%%%%%%%%%%%%%%%%%%%% }

%%%%%%%%%%%%%%%%%%%%%%%% CAMBIAR ESTO %%%%%%%%%%%%%%%%%%%% 
%%%%%%%%%%%%%%%%%%%%%%%%%%%%%%%%%%%%%%%%%%%%%%%%%%%%%%%%%% {
\newcommand\studentName{Your Name}            % <-- Nombre
\newcommand\studentEmail{email@gmail.com}    % <-- Correo
\newcommand\studentNumber{XXXXXXXX}          % <-- Cedula
%%%%%%%%%%%%%%%%%%%%%%%%%%%%%%%%%%%%%%%%%%%%%%%%%%%%%%%%%% }
%%%%%%%%%%%%%%%%%%%%%%%%%%%%%%%%%%%%%%%%%%%%%%%%%%%%%%%%%%

%%%%%%%%%%%%%%%%% NO CAMBIAR ESTA PARTE %%%%%%%%%%%%%%%%%%
%%%%%%%%%%%%%%%%%%%%%%%%%%%%%%%%%%%%%%%%%%%%%%%%%%%%%%%%%% {

\newcommand{\customBlock}[1]{\input{#1/\studentNumber}}

    \ShortHeadings{\university  ~~-  \course ~~- \courseName}{\studentName - \studentNumber}
    \firstpageno{1}
    
    \begin{document}
    
    \title{\evaluation ~ \assignmentNumber}
    
    \author{\name \studentName \email \studentEmail \\
    \studentNumber
    \addr
    }
    
    \maketitle
%%%%%%%%%%%%%%%%%%%%%%%%%%%%%%%%%%%%%%%%%%%%%%%%%%%%%%%%%%
%%%%%%%%%%%%%%%%%%%%%%%%%%%%%%%%%%%%%%%%%%%%%%%%%%%%%%%%%% }



%%%%%%%%%%%%%%%%% CONTENIDO DEL PARCIAL %%%%%%%%%%%%%%%%%%
%%%%%%%%%%%%%%%%%%%%%%%%%%%%%%%%%%%%%%%%%%%%%%%%%%%%%%%%%% {

%%ENUNCIADO
En este examen parcial, se desea que usted ponga en pr\'actica y exprese su comprensi\'on de los temas tratados hasta ahora (\href{https://github.com/ucvia/dm/blob/main/nota_informativa.pdf}{Temas 1 y 2}). El examen consta de una secci\'on te\'orica y una pr\'actica, en ambas, puede acceder a los recursos mencionados durante las clases siempre que indique su uso y argumente de manera apropiada sus respuestas. \textbf{Cualquier evidencia de plagio ser\'a penalizado con la m\'inima nota}.

\vskip 0.2in
\textbf{Instrucciones:}
\begin{itemize}
    \item \textbf{Modifique las lineas de 61, 62 y 63 del c\'odigo latex para colocar sus datos.}
    \item Modifique los archivos en la carpeta respuestas para inclu\'ir su respuesta a cada pregunta usando el mismo \'indice como nombre, ej: \textsl{respuestas/1.tex} ser\'a asumida c\'omo la respuesta a la pregunta 1.
    \item Envie el proyecto en formato comprimido (zip) con el patr\'on de nombre \textbf{p1\_x} (d\'onde x se refiere a su n\'umero de c\'edula) al correo \href{mailto:will.all.gs@gmail.com}{will.all.gs@gmail.com}
\end{itemize}

% \vskip 0.3in
\section*{Teoría}
\label{sec:theory}

\textbf{(2pts) Pregunta 1:} Suponga un conjunto de datos en su poder que fu\'e resultado de un proceso de etiquetado por un tercero y le permite modelar la variable objetivo aprobado ; $aprobado(i) \in (si, no)$ dónde $i$ se refiere a una observaci\'on. ¿Qué tipo de tarea (o tareas) podría realizar sobre dicho conjunto de datos?

\textbf{Respuesta: } {\color{ucvia}Respuesta 1}
\vskip 0.3in


\textbf{(2pts) Pregunta 2:} ¿Son las tareas de agrupaci\'on y asociaci\'on iguales? Argumente su respuesta.

\textbf{Respuesta: } {\color{ucvia}Respuesta 2}
\vskip 0.3in


\textbf{(2pts) Pregunta 3:} ¿Es posible utilizar la metodolog\'ia SEMMA en un entorno laboral tecnol\'ogico actual? Argumente su respuesta.

\textbf{Respuesta: } {\color{ucvia}Respuesta 3}
\vskip 0.3in

\textbf{(2pts) Pregunta 4:} Describa la relación entre la minería de datos y el aprendizaje automático.

\textbf{Respuesta: } {\color{ucvia}Respuesta 4}
\vskip 0.3in

% \vskip 0.5in
\newpage
\section*{Práctica}
\label{sec:practice}

\textbf{(3pts) Pregunta 5:} En el \href{https://www.youtube.com/watch?v=f1J38FlDKxo}{reciente evento de Apple} se anunci\'o el nuevo Ipad Pro, sugiera c\'omo las tareas de miner\'ia de datos podr\'ian ayudar a dicha compa\~n\'ia a medir el \'exito de su lanzamiento.

\textbf{Respuesta: } {\color{ucvia}Respuesta 5}
\vskip 0.3in

\textbf{(3pts) Pregunta 6:} Recientemente, dos artistas estadounidenses de m\'usica urbana, intercambiaron canciones desprestigiandose mutuamente ¿C\'omo la miner\'ia de datos puede ser usada para analizar la disputa?

\textbf{Respuesta: } {\color{ucvia}Respuesta 6}
\vskip 0.3in

\textbf{(3pts) Pregunta 7:} Utilizando \'areas de su inter\'es, proponga 3 conjuntos de datos y 2 posibles resultados de hacer miner\'ia de datos en cada uno, que ejemplifiquen el objetivo de la miner\'ia de datos. Use la nomenclatura para \textsl{basic data patterns} detallada en \cite{Ma2021MetaInsightAD}, \href{https://www.microsoft.com/en-us/research/uploads/prod/2016/12/Insight-Types-Specification.pdf}{\textsl{Insights type spec}} donde aplique.

\textbf{Respuesta: } {\color{ucvia}Respuesta 7}
\vskip 0.3in

\textbf{(3pts) Pregunta 8:} Usando los datos \customBlock{p2}.
Indique posibles aspectos resaltantes a considerar si quisiera realizar un proceso de minería de datos en un entorno real sobre estos datos usando las etapas de las metodologías vistas en clase.

\textbf{Respuesta: } {\color{ucvia}Respuesta 8}
\vskip 0.3in

\textbf{(3pts) Pregunta 9:} Usando los datos \customBlock{p2}.
Indique, con la nomenclatura vista en clase, posible conocimiento a encontrar usando solamente los nombres de las variables y sus descripciones.

\textbf{Respuesta: } {\color{ucvia}Respuesta 9}
\vskip 0.3in

\textbf{(3pts) Pregunta 10:} Escoga uno de los conjuntos de datos entre \customBlock{p2} y escoga un notebook que haya realizado EDA sobre los datos y opine sobre al menos 4 operaciones realizadas considerando, como en clase, calidad de visualizaciones en asistir la obtención de conocimiento entre otros criterios, argumente su opinión en cada caso.

\textbf{Respuesta: } {\color{ucvia}Respuesta 10}
\vskip 0.3in

\newpage
\textbf{(4pts) Pregunta 11:} En los siguientes casos especifique y justifique la tarea (o las tareas) de miner\'ia de datos que sugiere:
\begin{itemize}
    \item Un amigo le consulta una manera de estimar el precio de un inmueble dado su criterio de b\'usqueda.
    \item Desea conocer nuevas serias a ver contando con su historial de shows vistos.
    \item Desea comparar el tema entre las canciones que le gustan de un artista y las que no.
    \item Desea conocer cual es su g\'enero de m\'usica m\'as escuchado.
    \item Desea encontrar un monto cargado a su cuenta que desconoce (puede asumir una detallada representacion de cada transacci\'on).
\end{itemize}

\textbf{Respuesta: } {\color{ucvia}Respuesta 11}
\vskip 0.3in
%% HERE IS WHERE THE REFERENCES IS INCLUDED
%% TO ADD NEW ONES, GO TO THE FILE references.bib IN THE REFERENCES FOLDER AND ADD IN bibtex FORMAT
\vskip 0.5in
\bibliography{references}
%%%%%%%%%%%%%%%%%%%%%%%%%%%%%%%%%%%%%%%%%%%%%%%%%%%%%%%%%%
%%%%%%%%%%%%%%%%%%%%%%%%%%%%%%%%%%%%%%%%%%%%%%%%%%%%%%%%%% }
\end{document}
